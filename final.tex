\documentclass[11pt]{report}

\usepackage[margin=1in]{geometry} 
\usepackage{amsmath,amsthm,amssymb,amsfonts}

\usepackage[utf8]{inputenc}
\usepackage[T1]{fontenc}
\usepackage[spanish]{babel}


\usepackage{mathpazo,euler}
\linespread{1.05}
\usepackage{mathrsfs}

\usepackage{enumitem}
\usepackage{microtype}
\usepackage{tocbibind}

\usepackage{thmtools,xcolor}
\usepackage{fancyhdr}
\pagestyle{fancy}
\usepackage[misc]{ifsym}
\usepackage{tcolorbox}
\tcbuselibrary{theorems}

\usepackage{tikz}
\usepackage{tikz-cd}
\usetikzlibrary{arrows}
\usetikzlibrary{matrix}

\definecolor{color}{RGB}{0, 131, 126}
\addto\captionsspanish{\renewcommand{\chaptername}{Parte}}
\renewcommand\qedsymbol{$\tpaint{\small{\FilledSectioningDiamond}}$}
\declaretheoremstyle[
  headfont=\color{color}\normalfont\bfseries,
  notefont=\color{color}\normalfont\bfseries
]{colored}
\theoremstyle{colored}
\newtheorem{definition}{Definición}[section]
\newtheorem{theorem}{Teorema}[section]
\newtheorem*{theorem*}{Teorema}
\newtheorem{proposition}{Proposición}[section]
\newtheorem{corollary}{Corolario}[section]
\newtheorem{lemma}{Lema}[section]
\newtheorem{remark}{Observación}[section]
\newtheorem{example}{Ejemplo}[section]

\newcommand{\N}{\mathbb{N}}
\newcommand{\Z}{\mathbb{Z}}
\newcommand{\Q}{\mathbb{Q}}
\newcommand{\R}{\mathbb{R}}
\newcommand{\C}{\mathbb{C}}
\newcommand{\M}[2]{\mathsf{M}_{#1}#2}
\newcommand{\im}{\operatorname{im}}
\newcommand{\id}{\operatorname{id}}
\newcommand{\eps}{\varepsilon}
\newcommand{\nat}[1]{[\![#1]\!]}
\newcommand{\ord}[1]{\nat{#1}}
\newcommand{\natzero}[1]{\nat{#1}_0}
\newcommand{\ol}{\overline}
\newcommand{\cat}[1]{\mathsf{#1}}
\renewcommand{\ss}[1]{\Delta^{#1}}
\newcommand{\bss}[1]{\partial \ss{n}}
\newcommand{\horn}[2]{\Lambda^{#1}_{#2}}
\newcommand{\ordcat}{\boldsymbol{\Delta}}
\newcommand{\catlim}[2]{\underset{#1}{\operatorname{lim}}#2}
\newcommand{\catcolim}[2]{\underset{#1}{\operatorname{colim}}#2}

\newcommand{\paint}[1]{\color{color}{#1}}
\newcommand{\tpaint}[1]{\paint{\textbf{#1}}}
\newcommand{\paintline}{\begin{center}
$\paint{
\rule{400pt}{0.5pt}
}$
\vspace{10pt}
\end{center}}

%-----------------------

\title{
\LARGE{Topología Algebraica}
\\
\vspace{3pt}
\small{Primer Cuatrimestre -- 2019}
\\
\vspace{0.5pt}
\large{Examen Final}
\\
\vspace{80pt}
{\includegraphics[height=5cm]{uba2.jpg}}
\vspace{80pt}
}
\author{Guido Arnone}
\date{}
\lhead{Guido Arnone}
\rhead{Examen Final}

\begin{document}

\maketitle
\tableofcontents

\chapter{Preliminares}

\section{La categoría de ordinales finitos}

\begin{definition} Se define la \textbf{categoría $\ordcat$ de ordinales finitos} como la categoría que tiene por objetos a los conjuntos ordenados
\[
\ord{n} := \{0 < 1 < \cdots < n\}
\]
para cada $n \in \N_0$, y cuyas flechas son las funciones $f : \ord{n} \to \ord{m}$ que resultan morfismos de posets. Definimos además, para cada $\ord{n} \in \ordcat$ e $i \in \natzero{n}$, 
\begin{itemize}
\item los mapas de \textbf{cocaras},
\begin{align*}
d^i : \ord{n-&1} \to \ord{n}\\
&j \mapsto \begin{cases}
j &\text{si $j < i$}\\
j+1 &\text{si $j \geq i$}
\end{cases}
\end{align*}
y
\item los mapas de \textbf{codegeneraciones},
\begin{align*}
s^i : \ord{n+&1} \to \ord{n}\\
&j \mapsto \begin{cases}
j &\text{si $j \leq i$}\\
j-1 &\text{si $j > i$}
\end{cases}
\end{align*}
\end{itemize} 
\end{definition}

\begin{proposition} Los mapas de cocaras y codegeneraciones satisfacen las siguientes \textit{identidades cosimpliciales},
\[
\begin{cases}
d^jd^i = d^id^{j-1} &\text{si $i < j$}\\
s^jd^i = d^is^{j-1} &\text{si $i < j$}\\
s^jd^j = s^jd^{j+1} = 1 &\text{si $i < j$}\\
s^jd^i = d^{i-1}s^j &\text{si $i > j+1$}\\
s^js^i = s^is^{j+1} &\text{si $i \leq j$}\\
\end{cases}
\]
\qed \\
\end{proposition}

\begin{proposition} Toda flecha $\ord{n} \xrightarrow{f} \ord{m}$ en $\ordcat$ se puede escribir como una composición de mapas de cocaras y codegeneraciones. \\ \qed
\end{proposition}

\begin{remark} En vista de las dos proposiciones anteriores,  usando los mapas de cocaras y codegeneraciones y las identidades cosimpliciales se puede dar «una presentación de $\ordcat$ en términos de generadores y relaciones». A grandes rasgos, esto nos permitirá definir los objetos relacionados a $\ordcat$ únicamente a partir de los mapas de cocaras y codegeneraciones.
\end{remark}

\section{Conjuntos Simpliciales}

Ahora sí, pasamos a definir los conjuntos simpliciales:

\begin{definition} Un \textbf{conjunto simplcial} es un funtor $X : \ordcat^{op} \to \cat{Set}$. Concretamente, éste consiste de 
\begin{itemize}
\item[(i)] una sucesión $X_0,X_1,X_2, \dots$ de conjuntos, y \item[(ii)] para cada $n \in \N_0$ e $i \in \natzero{n}$, funciones 
$d_i : X_n \to X_{n-1}$ y $s_i : X_n \to X_{n+1}$ llamadas mapas de caras y degeneraciones respectivamente, que satisfacen las siguientes \textit{identidades simpliciales}:
\[
\begin{cases}
d_id_j = d_{j-1}d_i &\text{si $i < j$}\\
d_is_j = s_{j-1}d_i &\text{si $i < j$}\\
d_js_j = d_{j+1}s_j = 1 &\text{si $i < j$}\\
d_is_j = s_jd_{i-1} &\text{si $i > j+1$}\\
s_is_j = s_{j+1}s_i &\text{si $i \leq j$}\\
\end{cases}
\] 
\end{itemize}
\end{definition}

\begin{example}[el $n$-símplex estándar] Para cada $n \in \N_0$ tenemos un conjunto simplicial dado por el funtor $\ordcat(-,\ord{n})$. Conrectamente, para cada $j \geq 0$ definimos los conjuntos
\[
\ss{n}_m := \ordcat(\ord{m},\ord{n}) = \{f : \ord{m} \to \ord{n} : \text{ $f$ es morfismo de posets } \}
\]
y los mapas de caras y degeneraciones están dados por
\[
(d^i)^* : f \in \ordcat(\ord{m},\ord{n}) \mapsto fd^i \in \ordcat(\ord{m-1},\ord{n})
\]
y
\[
(s^i)^* : f \in \ordcat(\ord{m},\ord{n}) \mapsto fs^i \in \ordcat(\ord{m+1},\ord{n}).
\]
Llamamos a este conjunto simplicial el \textbf{n-símplex estándar} y lo notamos $\ss{n}$.
\end{example}

Una vez más interpretando a $\ord{n}$ como el $n$-símplex combinatorio ordenado, su conjunto simplicial asociado consiste de «todas las formas posibles de incluir o colapsar un $m$-símplex estándar en $\nat{n}$».

Extendiendo esta interpretación tenemos el siguiente ejemplo,

\begin{example}[complejos simpliciales ordenados] Sea $K$ un complejo simplicial equipado con una relación de orden total para sus vértices $V = \{v_i\}_{i \in I}$. Notamos a cada $n$-símplex como una $n$-upla $[v_{i_0},\dots, v_{i_n}]$ con $v_k < v_{k+1}$ para cada $k$.

Asociaremos a $K$ un conjunto simplicial, agregando como en el caso de $\ss{n}$ la noción de \textit{símlpices degenerados}. Concretamente, para cada $n \in \N_0$ definimos
\[
K_n := \left\{[v_{i_0}, \dots, v_{i_n}] : \text{ $v_{i_k} \leq v_{i_{k+1}}$ para cada $k$, y $\{v_{i_k}\}_{k = 0}^n \in K$ }\right\}.
\]

En otras palabras, el conjunto $K_n$ consiste de $n$-uplas ordenadas de vértices que forman un símplex de $K$, pero permitiendo repetición. Definimos a su vez los mapas de caras y degeneraciones como
\[
d_k[v_{i_0}, \dots, v_{i_n}] := [v_{i_0},\dots,\widehat{v_{i_k}},\dots, v_{i_n}] \quad \text{y} \quad
s_k[v_{i_0}, \dots, v_{i_n}] := [v_{i_0},\dots,v_{i_k},v_{i_k}, \dots, v_{i_n}].
\]
\end{example}

\begin{definition} Dado un $n$-śimplex estándar (como complejo simplicial), su \textbf{borde} es el complejo $\bss{n}$ dado por la unión de sus caras maximalesy su $k$-ésimo \textbf{cuerno} es el subcomplejo $\horn{n}{k}$ de $\bss{n}$ que se obtiene quitando la $k$-ésima cara maximal de $\ss{n}$, para cierto $0 \leq k \leq n$. Decimos que $\horn{n}{k}$ es un cuerno \textit{interno} si $0 < k < n$, y \textit{externo} en caso contrario.
\end{definition}

Ahora sí, veamos un primer ejemplo topológico:

\begin{example} Sea $X$ un espacio topológico. Para cada $n \in \N_0$ definimos el conjunto
\[
\mathcal{S}(X)_n := \cat{Top}(|\ss{n}|,X)
\]
de todos los $n$-simplices singulares de $X$, y las aplicaciones 
\[
d_i : \mathcal{S}(X)_n \to \mathcal{S}(X)_{n-1}, \quad
s_i : \mathcal{S}(X)_n \to \mathcal{S}(X)_{n+1}
\]
que envían un $n$-símplex singular a la restricción $d_i\sigma$ a su $i$-ésima cara y al $(n+1)$-simplex singular $s_i\sigma$ que corresponde a colapsar $|\ss{n+1}|$ a $|\ss{n}|$ a través de $|s^i|$ y luego componer con $\sigma$.

Estos conforman el conjunto simplicial $\mathcal{S}(X)$ que se conoce como el \textbf{conjunto singular} de $X$.
\end{example}

\begin{definition} Dado un conjunto simplicial $X$, definimos su \textbf{complejo de Moore} como el complejo de cadenas
\[
\cdots \to \Z X_2 \xrightarrow{\partial} \Z X_1 \xrightarrow{\partial} \Z X_0,
\]
con $\Z X_n$ el grupo abeliano libre generado por el conjunto $X_n$ y 
\[
\partial = \sum_{i=1}^n(-1)^i d_i
\]
para cada $n \geq 0$.
\end{definition}

\begin{remark} La homología singular de un espacio topológico $X$ coincide con la homología del complejo de Moore de su conjunto singular.
\end{remark}

\begin{example}[nervio de una categoría] Sea $\mathscr{C}$ una categoría localmente pequeña. Definimos el \textbf{nervio} de  $\mathscr{C}$ como el conjunto simplicial dado por los conjuntos
\begin{align*}
N(\mathscr{C})_n = \hom(\mathbf{n},\mathscr{C}) &= \{(f_1,\dots, f_k) : f_i \in \operatorname{mor} \mathscr{C}, \ \operatorname{cod} f_i = \operatorname{dom}f_{i+1}\}\\
& = \{x_0 \xrightarrow{f_1} x_1 \xrightarrow{f_2} \cdots \xrightarrow{f_n} x_n\}
\end{align*}
de $n$-uplas de morfismos componibles junto con los mapas 
\[
d_i(f_1, \dots, f_n) = (f_1, \dots,f_{i-1}, \ f_i \circ f_{i+1} \ ,\dots,f_n)
\]
y
\[
s_i(f_1, \dots, f_n) = (f_1, \dots, f_i, 1, f_{i+1},\dots,f_n),
\]
para cada $0 \leq i \leq n$. 

En particular, si $\mathscr{C} = BG$ es el grupoide asociado a un grupo $G$, entonces su nervio consiste de $n$-uplas de elementos de $G$ y los mapas de caras y degeneraciones están dados por
\[
d_i(g_1, \dots, g_n) = (g_1,\dots, g_{i-1},g_ig_{i+1},\dots,g_n), \quad s_i(g_1, \dots,g_n) = (g_1, \dots,g_i,1,g_{i+1},\dots,g_n).
\]
\end{example}

\begin{definition} Dados dos complejos simpliciales $X,Y : \ordcat^{op} \to \cat{Set}$, un \textbf{morfismo de conjuntos simpliciales} de $X$ a $Y$ es una transformación natural $f : X \to Y$. Concretamente, esto consiste en dar una familia de funciones $f_n : X_n \to Y_n$ tales que, para cada $0 \leq i \leq n$, los siguientes diagramas conmutan
\begin{center}
\begin{tikzpicture}
\matrix (m) [matrix of math nodes,row sep=3em,column sep=4em,minimum width=2em]
  {
     X_n & X_{n-1} & & X_n & X_{n+1}\\
     Y_n & Y_{n-1} & & Y_n & Y_{n+1}\\};
  \path[-stealth]
    (m-1-1) edge node [left] {$f_n$} (m-2-1)
    (m-1-4) edge node [left] {$f_n$} (m-2-4)
    (m-1-2) edge node [right] {$f_{n-1}$} (m-2-2)
    (m-1-5) edge node [right] {$f_{n+1}$} (m-2-5)
    (m-1-1) edge node [above] {$d_i$} (m-1-2)
    (m-2-1) edge node [below] {$d_i$} (m-2-2)
    (m-1-4) edge node [above] {$s_i$} (m-1-5)
    (m-2-4) edge node [below] {$s_i$} (m-2-5);
\end{tikzpicture} 
\end{center}

Es decir, un morfismo de conjuntos simpliciales consiste de una colección de aplicaciones que sea compatible con los mapas de caras y degeneraciones.
\end{definition}

\begin{remark} Los conjuntos simpliciales junto con los morfismos antes definidos forman una categoría\footnote{Esta es precisamente la categoría de prehaces de $\ordcat^{op}$.}, que notaremos $\cat{sSet}$.
\end{remark}

\begin{remark} Un morfismo $f : K \to L$ de complejos simlpiciales ordenados induce a su vez un morfismo de conjuntos simpliciales dado por $f_n[v_{i_1}, \dots, v_{i_n}] := [f(v_{i_1}), \dots, f(v_{i_n})]$ para cada $n$-símplex $[v_{i_1}, \dots, v_{i_n}]$ (posiblemente degenerado) de $K$.

De forma similar, una función continua $f : X \to Y$ induce un morfismo $f_* : \mathcal{S}(X) \to \mathcal{S}(Y)$ entre conjuntos singulares vía la postcomposición. De hecho, 
\end{remark}

\begin{definition} El \textbf{funtor singular} $\mathcal{S} : \cat{Top} \to \cat{sSet}$ asigna a cada espacio su conjunto singular, y a cada función continua $f : X \to Y$ el morfismo simplicial $f_* : \mathcal{S}(X) \to \mathcal{S}(Y)$ dado por $(f_*)_n(\sigma) =  f \circ \sigma$ para cada $\sigma: |\ss{n}| \to X$.
\end{definition}

\begin{remark} Si $X$ es un conjunto simplicial, el conjunto $X_n$ está determinado por los morfismos de conjuntos simpliciales de $\ss{n}$ a $X$. 

Concretamente, por el lema de Yoneda tenemos una biyección natural
\[
\hom_{\cat{sSet}}(\ss{n},X) \simeq X_n
\]

que a cada elemento $x \in X_n$ le asigna un morfismo de conjuntos simpliciales $\iota_x : \ss{n} \to X$ que satisface $\iota_x(1_{\ord{n}}) = x$. 

Esto se corresponde con la intuición que proveen los ejemplos anteriores, en los que los $n$-símplices de un conjunto simplicial son alguna «manifestación» de el $n$-símplex estándar: como la cara de un $m$-símplex de dimensión mayor, como un $m$-símplex de dimensión menor que represente un colapso del mismo, o como el $n$-símplex singular de un espacio topológico.\\
\end{remark}

\begin{definition} Sea $X$ un conjunto simplicial. Un $n$-simplex $x \in X_n$ se dice \textbf{degenerado} si existe $y \in X_{n-1}$ tal que $s_i(y) = x$ para algún $i \in \natzero{n}$. En caso contrario, decimos que $x$ es \textbf{no degenerado}.
\end{definition}

\section{Realización Geométrica}

Durante la materia vimos como a partir de un complejo simplicial $K$ podemos construir un espacio topológico $|K|$, la realización geométrica de $K$. Siguiendo esta idea, queremos extender esta noción al contexto de los conjuntos simpliciales.

\begin{definition}[realización geométrica, primera definición] Sea $X$ un conjunto simplicial. Dotando a cada conjunto $X_n$ de la topología discreta, definimos la \textbf{realización geométrica} de $X$ como el espacio
\[
|X| = \left(\coprod_{n \geq 0}X_n \times |\ss{n}|\right)\Big/\sim
\]
donde identificamos a los puntos $(x,|d^i|(p)) \sim (d_i(x),p)$ y $(x,|s^i|(p)) \sim (s_i(x),p)$.
\end{definition}

Esto formaliza la intuición anterior: para cada $x \in X_n$ construimos una copia del $n$-símplex estándar y los pegamos en función de si son caras o colapsos unos de otros.

Además, como veremos en breve esta asignación es funtorial, teniéndose así un análogo a la realización geométrica para complejos simpliciales.

\begin{proposition} La realización geométrica del $n$-símplex estándar $\ss{n} :\ordcat^{op} \to \cat{Set}$ es homeomorfa a la realización geométrica del $n$-símplex estándar como complejo simplicial.
\end{proposition}
\begin{proof} Para evitar ambiguedades, en toda la demostración notaremos $|\ss{n}|$ exclusivamente para referirnos a la realización geométrica del $n$-símplex como complejo simplicial. Por otro lado, notaremos $|\ordcat(-,n)|$.

Para cada $k \in \N_0$ tenemos una aplicación
\[
(f,x) \in \ordcat(k,n) \times |\Delta^k| \to |f|(x) \in |\Delta^n|,
\]
que resulta continua pues cada espacio $\ordcat(k,n)$ es discreto.

Ésta familia de funciones induce un morfismo en el coproducto que pasa al cociente por las identificaciones de la realización geométrica, pues si $g$ es un mapa de cocara o codegeneración, entonces
\[
r(g^*(f),x) = |fg|(x) = r(f,|g|(x)).
\]
Se tiene entonces una función continua $r : [(f,x)] \in |\ordcat(-,n)| \to |f|(x) \in |\Delta^n|$. 

Por otro lado, podemos considerar la inclusión $i : |\ss{n}| \to |\ordcat(-,n)|$ dada por la composición
\[
|\ss{n}| \xrightarrow{\simeq} \{id\} \times |\ss{n}| \hookrightarrow |\ordcat(-,n)|,
\]
que satisface $ri = 1$ pues
\[
ri(x) = [(id,x)] = |id|(x) = x
\]
para todo $x \in |\ss{n}|$. En particular sabemos que $i$ es inyectiva. Por lo tanto, para terminar basta ver que $i$ es sobreyectiva, en cuyo caso es un homeomorfismo con inversa $f$.

Equivalentemente, resta ver que todo elemento $[(f,x)]$ está relacionado con un punto de la forma $(id,y)$ para cierto $y \in |\ss{n}|$. En efecto, sea $(f,x) \in |\ordcat(-,n)|$ con $f : \ord{k} \to \ord{n}$ un morfismo de posets y $x \in |\ss{k}|$. Sabemos entonces que existen mapas de cocara o codegeneración $f_1, \dots, f_n$ tales que $f = f_1 \cdots f_n$. En consecuencia es
\begin{align*}
[(f,x)] &= [(f_1 \cdots f_n, x)] = [(f_n^* \circ \cdots \circ f_1^*(id), x)]\\ &= [(id,|f_1 \cdots f_n|(x))] = [(id, |f|(x))],
\end{align*}
lo que concluye la demostración.
\end{proof}

\begin{definition} Sea $X$ un conjunto simplicial. Su \textbf{categoría de símplices} es la categoría coma $\Delta \downarrow X$, donde $\Delta: \ordcat \to \cat{Set}^{\ordcat^{op}}$ es el embedding de Yoneda de $\ordcat$ y $X : \ordcat \to \cat{Set}^{\ordcat^{op}}$ es el funtor que vale constantemente $X$. Concretamente, los objetos de $\Delta \downarrow X$ son \textit{símplices} de $X$, entendidos como morfismos simpliciales $\ss{n} \to X$, y las flechas son morfismos $\theta : \ord{n} \to \ord{m}$ de posets tales que el diagrama
\begin{center}
\begin{tikzpicture}
\matrix (m) [matrix of math nodes,row sep=3em,column sep=4em,minimum width=2em]
  {
     \ss{n}& &\\
     & & X\\
     \ss{m} & &\\};
  \path[-stealth]
    (m-1-1) edge node [above] {$\sigma$} (m-2-3)
    (m-1-1) edge node [left] {$\theta_*$} (m-3-1)
    (m-3-1) edge node [below] {$\tau$} (m-2-3);
\end{tikzpicture} 
\end{center}
conmuta, donde $\theta_*$ es la postcomposición por $\theta$.
\end{definition}

\begin{theorem} Si $X$ es un conjunto simplicial, entonces
\[
|X| \simeq \catcolim{\substack{\ss{n} \to X \\ \text{ en $\Delta \downarrow X$}}}{|\ss{n}|}.
\]
\end{theorem}
\begin{proof} Observemos que si $Z$ es un espacio topológico arbitrario, una función continua $g : |X| \to Z$ se corresponde unívocamente con una función continua $\tilde{g} : \coprod_{n \geq 0} X_n \times |\ss{n}| \to Z$ que sea compatible con la relación que identifica caras y colapsos.

A su vez, usando la propiedad universal del coproducto y el hecho de que cada espacio $X_n$ tiene la topología discreta, esto equivale a dar funciones
\[
\lambda_x : |\ss{n}| \to Z
\]
para cada $x \in X_n$ y $n \geq 0$, que satisfagan las condiciones de compatibilidad de antes: esto es, que para cada morfismo de posets\footnote{Si bien la condición original era sobre los mapas de caras y degeneración, que son imagen por $X$ de los mapas de cocaras y codegeneración en $\ordcat$. Al clausurar la relación de equivalencia se tiene una condición equivalente reemplazando éstos por cualquier función $X(\theta)$ con $\theta$ un morfismo en $\ordcat$.} $\theta : \ord{n} \to \ord{m}$ se tenga $\lambda_{X(\theta)(x)} = \lambda_x \circ |\theta| = |\theta_*|(\lambda_x)$.

Recordemos ahora que, por el lema de Yoneda, tenemos que
\[
\hom(\ss{n},\ss{m}) \simeq \ordcat(n,m) = \{\theta : \ord{n} \to \ord{m} : \text{ $\theta$ es morfismo de posets}\}
\]
y
\[
\hom(\ss{n}, X) \simeq X_n
\]
para cada $n,m \geq 0$. Por lo tanto, cada elemento $x \in X_n$ se corresponde a un único morfismo simplicial $\sigma_x : \ss{n} \to X$, y toda flecha $\ss{n} \to \ss{m}$ es la poscomposición por cierto morfismo de posets $\theta: \ord{n} \to \ord{m}$.

En éstos terminos, la información anterior se puede describir como un morfismo $\lambda_\sigma : |\ss{n}| \to Z$ para cada símplex $\sigma : \ss{n} \to X$, de forma que para todo morfismo de posets $\theta : \ord{n} \to \ord{m}$ y símplices $\sigma : \ss{n} \to X, \tau : \ss{m} \to X$ tales que $\sigma = \tau \  \theta_*$ se tenga que $\lambda_\sigma = \lambda_\tau |\theta_*|$. 

Es decir, dar un morfismo $g : |X| \to Z$ es equivalente a dar un cocono sobre $Z$ para el funtor $F : \Delta \downarrow X \to \mathsf{Top}$ que envía $(\sigma : \ss{n} \to X) \xrightarrow{\theta_*} (\tau : \ss{m} \to X)$ a $|\ss{n}| \xrightarrow{|\theta_*|} |\ss{m}|$.

Por otro lado, tenemos un cocono sobre $X$ dado por las funciones
\[
\iota_\sigma : p \in |\ss{n}| \to [(x,p)] \in |X|,
\]
para cada $\sigma : \ss{n} \to X$ que está en correspondencia con $x \in X_n$. Por las observaciones anteriores, sabemos que $(\lambda_\sigma)_{\sigma \in \Delta \downarrow X}$ se factoriza por $(\iota_\sigma)_{\sigma \in \Delta \downarrow X}$ a través de $g$ ya que es
\[
\lambda_\sigma(p) = g([(x,p)]) = g(\iota_\sigma(x,p))
\]
para todo $\sigma : \ss{n} \to X$ y $p \in |\ss{n}|$. Pero como notamos anteriormente, de existir una función continua $|X| \to Z$ está determinada por lo que vale en la imagen de cada morfismo $\iota_\sigma$. 
Hemos visto entonces que todo cono sobre $F : \Delta \downarrow X \to \mathsf{Top}$ se factoriza a través de $(\iota_\sigma)_\sigma$ de forma única, y esto es precisamente que
\[
|X| \simeq \catcolim{\substack{\ss{n} \to X \\ \text{ en $\Delta \downarrow X$}}}{|\ss{n}|}.
\]
\end{proof}

\begin{proposition} Se tiene un funtor
\[
| \cdot | : \cat{sSet} \to \cat{Top}
\]
que asigna a cada conjunto simplicial $X$ su realización geométrica, y a cada morfismo simplicial $f : X \to Y$ una flecha $|f| : |X| \to |Y|$ inducida por cada función $f_n \times 1 : X_n \times |\ss{n}| \to Y_n \times |\ss{n}|$.
\end{proposition}
\begin{proof} En primer lugar, notemos que las aplicaciones $\{f_n \times 1\}_{n \geq 1}$ inducen una función continua
\begin{center}
\begin{tikzcd}
 \coprod_{n \geq 0} X_n \times |\ss{n}| \arrow{r}{\widetilde{f}} &  \coprod_{n \geq 0} Y_n \times |\ss{n}|\\
 X_n \times |\ss{n}| \arrow[hookrightarrow]{u} \arrow{r}{f_n \times 1} & Y_n \times |\ss{n}| \arrow[hookrightarrow]{u}
\end{tikzcd}
\end{center}

entre los coproductos. Como $f$ es un morfismo simplicial, si $\theta : \ord{m} \to \ord{m}$ es un morfismo de posets entonces el diagrama
\begin{center}
\begin{tikzcd}
 X_n \times |\ss{n}| \arrow{r}{X\theta}\arrow{d}[left]{f_n \times 1} &  X_m \times |\ss{n}| \arrow{d}{f_m \times 1}\\
 Y_n \times |\ss{n}| \arrow{r}{Y \theta} & Y_m \times |\ss{n}| 
\end{tikzcd}
\end{center}
conmuta. Por lo tanto, se tiene que
\[
\widetilde{f}(X\theta(x),p) = (f_mX\theta(x),p) = (Y\theta f_n(x),p)
\]
y
\[
\widetilde{f}(x,|\theta|(p)) = (f_n(x),|\theta|(p)),
\]
lo que nos dice que $\widetilde{f}$ manda puntos relacionados en puntos relacionados. En consecuencia, está bien definida la función continua $|f| : |X| \to |Y|$ que envía $[(x,p)]$ a $[(f_n(x),p)]$ si $x \in X_n$, y de esta caracterización se deduce la funtorialidad de $|\cdot|$.
\end{proof}

\begin{thebibliography}{}
\bibitem{GJ} P. Goerss y J. Jardine, \textit{Simplicial Homotopy Theory}. Birkhäuser, 2010.
\bibitem{F} G. Friedman. \textit{An elementary introduction to simplicial sets}, arXiv:0809.4221v5 [at], 2016.
\end{thebibliography}
\end{document}


