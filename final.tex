\documentclass[11pt]{report}

\usepackage[margin=1in]{geometry} 
\usepackage{amsmath,amsthm,amssymb,amsfonts}

\usepackage[utf8]{inputenc}
\usepackage[T1]{fontenc}
\usepackage[spanish]{babel}


\usepackage{mathpazo,euler}
\linespread{1.05}
\usepackage{mathrsfs}

\usepackage{enumitem}
\usepackage{microtype}
\usepackage{tocbibind}

\usepackage{thmtools,xcolor}
\usepackage{fancyhdr}
\pagestyle{fancy}
\usepackage[misc]{ifsym}
\usepackage{tcolorbox}
\tcbuselibrary{theorems}

\usepackage{tikz}
\usepackage{tikz-cd}
\usetikzlibrary{arrows}
\usetikzlibrary{matrix}

\definecolor{color}{RGB}{0, 131, 126}
\addto\captionsspanish{\renewcommand{\chaptername}{Parte}}
\renewcommand\qedsymbol{$\paint{\blacklozenge}$}
\declaretheoremstyle[
  headfont=\color{color}\normalfont\bfseries,
  notefont=\color{color}\normalfont\bfseries
]{colored}
\theoremstyle{colored}
\newtheorem{definition}{Definición}[section]
\newtheorem{theorem}{Teorema}[section]
\newtheorem*{theorem*}{Teorema}
\newtheorem{proposition}{Proposición}[section]
\newtheorem{corollary}{Corolario}[section]
\newtheorem{lemma}{Lema}[section]
\newtheorem{remark}{Observación}[section]
\newtheorem{example}{Ejemplo}[section]

\usepackage{hyperref}
\hypersetup{
    colorlinks,
    citecolor=color,
    filecolor=color,
    linkcolor=color,
    urlcolor=color,
}

\newcommand{\N}{\mathbb{N}}
\newcommand{\Z}{\mathbb{Z}}
\newcommand{\Q}{\mathbb{Q}}
\newcommand{\R}{\mathbb{R}}
\newcommand{\C}{\mathbb{C}}
\newcommand{\M}[2]{\mathsf{M}_{#1}#2}
\newcommand{\im}{\operatorname{im}}
\newcommand{\id}{\operatorname{id}}
\newcommand{\eps}{\varepsilon}
\newcommand{\nat}[1]{[\![#1]\!]}
\newcommand{\ord}[1]{\nat{#1}}
\newcommand{\natzero}[1]{\nat{#1}_0}
\newcommand{\ol}{\overline}
\newcommand{\tint}[1]{\stackrel{o}{#1}}
\newcommand*{\dt}[1]{\accentset{\mbox{\large\bfseries .}}{#1}}
\newcommand{\cat}[1]{\mathsf{#1}}
\newcommand{\sk}{\mathsf{sk}}
\renewcommand{\ss}[1]{\Delta^{#1}}
\newcommand{\bss}[1]{\partial \ss{n}}
\newcommand{\horn}[2]{\Lambda^{#1}_{#2}}
\newcommand{\ordcat}{\boldsymbol{\Delta}}
\newcommand{\catlim}[2]{\underset{#1}{\operatorname{lim}}#2}
\newcommand{\catcolim}[2]{\underset{#1}{\operatorname{colim}}#2}

\newcommand{\paint}[1]{\color{color}{#1}}
\newcommand{\tpaint}[1]{\paint{\textbf{#1}}}
\newcommand{\paintline}{\begin{center}
$\paint{
\rule{400pt}{0.5pt}
}$
\vspace{10pt}
\end{center}}

%-----------------------

\title{
\LARGE{Topología Algebraica}
\\
\vspace{3pt}
\small{Primer Cuatrimestre -- 2019}
\\
\vspace{0.5pt}
\large{Examen Final}
\\
\vspace{80pt}
{\includegraphics[height=5cm]{uba2.jpg}}
\vspace{80pt}
}
\author{Guido Arnone}
\date{}
\lhead{Guido Arnone}
\rhead{Examen Final}

\begin{document}

\maketitle
\tableofcontents

\chapter{Preliminares}

\section{La categoría de ordinales finitos}

\begin{definition} Se define la \textbf{categoría $\ordcat$ de ordinales finitos} como la categoría que tiene por objetos a los conjuntos ordenados
\[
\ord{n} := \{0 < 1 < \cdots < n\}
\]
para cada $n \in \N_0$, y cuyas flechas son las funciones $f : \ord{n} \to \ord{m}$ que resultan morfismos de posets. Definimos además, para cada $\ord{n} \in \ordcat$ e $i \in \natzero{n}$, 
\begin{itemize}
\item los mapas de \textbf{cocaras},
\begin{align*}
d^i : \ord{n-&1} \to \ord{n}\\
&j \mapsto \begin{cases}
j &\text{si $j < i$}\\
j+1 &\text{si $j \geq i$}
\end{cases}
\end{align*}
y
\item los mapas de \textbf{codegeneraciones},
\begin{align*}
s^i : \ord{n+&1} \to \ord{n}\\
&j \mapsto \begin{cases}
j &\text{si $j \leq i$}\\
j-1 &\text{si $j > i$}
\end{cases}
\end{align*}
\end{itemize} 
\end{definition}

\begin{proposition} Los mapas de cocaras y codegeneraciones satisfacen las siguientes \textit{identidades cosimpliciales},
\[
\begin{cases}
d^jd^i = d^id^{j-1} &\text{si $i < j$}\\
s^jd^i = d^is^{j-1} &\text{si $i < j$}\\
s^jd^j = s^jd^{j+1} = 1\\
s^jd^i = d^{i-1}s^j &\text{si $i > j+1$}\\
s^js^i = s^is^{j+1} &\text{si $i \leq j$}\\
\end{cases}
\]
\qed \\
\end{proposition}

\begin{proposition} Toda flecha $\ord{n} \xrightarrow{f} \ord{m}$ en $\ordcat$ se puede escribir como una composición de mapas de cocaras y codegeneraciones. \\ \qed
\end{proposition}

\begin{remark} En vista de las dos proposiciones anteriores,  usando los mapas de cocaras y codegeneraciones y las identidades cosimpliciales se puede dar «una presentación de $\ordcat$ en términos de generadores y relaciones». A grandes rasgos, esto nos permitirá definir los objetos relacionados a $\ordcat$ únicamente a partir de los mapas de cocaras y codegeneraciones.
\end{remark}

\section{Conjuntos Simpliciales}

Ahora sí, pasamos a definir los conjuntos simpliciales:

\begin{definition} Un \textbf{conjunto simplcial} es un funtor $X : \ordcat^{op} \to \cat{Set}$. Concretamente, éste consiste de 
\begin{itemize}
\item[(i)] una sucesión $X_0,X_1,X_2, \dots$ de conjuntos, y \item[(ii)] para cada $n \in \N_0$ e $i \in \natzero{n}$, funciones 
$d_i : X_n \to X_{n-1}$ y $s_i : X_n \to X_{n+1}$ llamadas mapas de caras y degeneraciones respectivamente, que satisfacen las siguientes \textit{identidades simpliciales}:
\[
\begin{cases}
d_id_j = d_{j-1}d_i &\text{si $i < j$}\\
d_is_j = s_{j-1}d_i &\text{si $i < j$}\\
d_js_j = d_{j+1}s_j = 1\\
d_is_j = s_jd_{i-1} &\text{si $i > j+1$}\\
s_is_j = s_{j+1}s_i &\text{si $i \leq j$}\\
\end{cases}
\] 
\end{itemize}
Notamos $x \in X$ si $x \in X_n$ para algún $n \in \N_0$, y decimos que $x$ es un \textbf{$n$-símplex generalizado}\footnote{Cuando no sea necesario aclararlo, diremos simplemente que $x$ es un símplex de $X$.} de $X$.\\
\end{definition}

\begin{example}[el $n$-símplex estándar] Para cada $n \in \N_0$ tenemos un conjunto simplicial dado por el funtor $\ordcat(-,\ord{n})$. Conrectamente, para cada $m \geq 0$ definimos los conjuntos
\[
\ss{n}_m := \ordcat(\ord{m},\ord{n}) = \{f : \ord{m} \to \ord{n} : \text{ $f$ es morfismo de posets } \}
\]
y los mapas de caras y degeneraciones están dados por
\[
(d^i)^* : f \in \ordcat(\ord{m},\ord{n}) \mapsto fd^i \in \ordcat(\ord{m-1},\ord{n})
\]
y
\[
(s^i)^* : f \in \ordcat(\ord{m},\ord{n}) \mapsto fs^i \in \ordcat(\ord{m+1},\ord{n}).
\]
Llamamos a este conjunto simplicial el \textbf{n-símplex estándar} y lo notamos $\ss{n}$.
\end{example}

Una vez más interpretando a $\ord{n}$ como el $n$-símplex combinatorio ordenado, su conjunto simplicial asociado consiste de «todas las formas posibles de incluir o colapsar un $m$-símplex estándar en $\nat{n}$».

Extendiendo esta interpretación tenemos el siguiente ejemplo,

\begin{example}[complejos simpliciales ordenados] Sea $K$ un complejo simplicial equipado con una relación de orden total para sus vértices $V = \{v_i\}_{i \in I}$. Notamos a cada $n$-símplex como una $n$-upla $[v_{i_0},\dots, v_{i_n}]$ con $v_k < v_{k+1}$ para cada $k$.

Asociaremos a $K$ un conjunto simplicial, agregando como en el caso de $\ss{n}$ la noción de \textit{símlpices degenerados}. Concretamente, para cada $n \in \N_0$ definimos
\[
K_n := \left\{[v_{i_0}, \dots, v_{i_n}] : \text{ $v_{i_k} \leq v_{i_{k+1}}$ para cada $k$, y $\{v_{i_k}\}_{k = 0}^n \in K$ }\right\}.
\]

En otras palabras, el conjunto $K_n$ consiste de $n$-uplas ordenadas de vértices que forman un símplex de $K$, pero permitiendo repetición. Definimos a su vez los mapas de caras y degeneraciones como
\[
d_k[v_{i_0}, \dots, v_{i_n}] := [v_{i_0},\dots,\widehat{v_{i_k}},\dots, v_{i_n}] \quad \text{y} \quad
s_k[v_{i_0}, \dots, v_{i_n}] := [v_{i_0},\dots,v_{i_k},v_{i_k}, \dots, v_{i_n}].
\]
\end{example}

\begin{definition} Dado un $n$-śimplex estándar (como complejo simplicial), su \textbf{borde} es el complejo $\bss{n}$ dado por la unión de sus caras maximalesy su $k$-ésimo \textbf{cuerno} es el subcomplejo $\horn{n}{k}$ de $\bss{n}$ que se obtiene quitando la $k$-ésima cara maximal de $\ss{n}$, para cierto $0 \leq k \leq n$. Decimos que $\horn{n}{k}$ es un cuerno \textit{interno} si $0 < k < n$, y \textit{externo} en caso contrario.
\end{definition}

Ahora sí, veamos un primer ejemplo topológico:

\begin{example} Sea $X$ un espacio topológico. Para cada $n \in \N_0$ definimos el conjunto
\[
\mathcal{S}(X)_n := \cat{Top}(|\ss{n}|,X)
\]
de todos los $n$-simplices singulares de $X$, y las aplicaciones 
\[
d_i : \mathcal{S}(X)_n \to \mathcal{S}(X)_{n-1}, \quad
s_i : \mathcal{S}(X)_n \to \mathcal{S}(X)_{n+1}
\]
que envían un $n$-símplex singular a la restricción $d_i\sigma$ a su $i$-ésima cara y al $(n+1)$-simplex singular $s_i\sigma$ que corresponde a colapsar $|\ss{n+1}|$ a $|\ss{n}|$ a través de $|s^i|$ y luego componer con $\sigma$.

Estos conforman el conjunto simplicial $\mathcal{S}(X)$ que se conoce como el \textbf{conjunto singular} de $X$.
\end{example}

\begin{definition} Dado un conjunto simplicial $X$, definimos su \textbf{complejo de Moore} como el complejo de cadenas
\[
\cdots \to \Z X_2 \xrightarrow{\partial} \Z X_1 \xrightarrow{\partial} \Z X_0,
\]
con $\Z X_n$ el grupo abeliano libre generado por el conjunto $X_n$ y 
\[
\partial = \sum_{i=1}^n(-1)^i d_i
\]
para cada $n \geq 0$.
\end{definition}

\begin{remark} La homología singular de un espacio topológico $X$ coincide con la homología del complejo de Moore de su conjunto singular.
\end{remark}

\begin{example}[nervio de una categoría] Sea $\mathscr{C}$ una categoría localmente pequeña. Definimos el \textbf{nervio} de  $\mathscr{C}$ como el conjunto simplicial dado por los conjuntos
\begin{align*}
N(\mathscr{C})_n = \hom(\mathbf{n},\mathscr{C}) &= \{(f_1,\dots, f_k) : f_i \in \operatorname{mor} \mathscr{C}, \ \operatorname{cod} f_i = \operatorname{dom}f_{i+1}\}\\
& = \{x_0 \xrightarrow{f_1} x_1 \xrightarrow{f_2} \cdots \xrightarrow{f_n} x_n\}
\end{align*}
de $n$-uplas de morfismos componibles junto con los mapas 
\[
d_i(f_1, \dots, f_n) = (f_1, \dots,f_{i-1}, \ f_i \circ f_{i+1} \ ,\dots,f_n)
\]
y
\[
s_i(f_1, \dots, f_n) = (f_1, \dots, f_i, 1, f_{i+1},\dots,f_n),
\]
para cada $0 \leq i \leq n$. 

En particular, si $\mathscr{C} = BG$ es el grupoide asociado a un grupo $G$, entonces su nervio consiste de $n$-uplas de elementos de $G$ y los mapas de caras y degeneraciones están dados por
\[
d_i(g_1, \dots, g_n) = (g_1,\dots, g_{i-1},g_ig_{i+1},\dots,g_n), \quad s_i(g_1, \dots,g_n) = (g_1, \dots,g_i,1,g_{i+1},\dots,g_n).
\]
\end{example}

\begin{definition} Dados dos complejos simpliciales $X,Y : \ordcat^{op} \to \cat{Set}$, un \textbf{morfismo de conjuntos simpliciales} de $X$ a $Y$ es una transformación natural $f : X \to Y$. Concretamente, esto consiste en dar una familia de funciones $f_n : X_n \to Y_n$ tales que, para cada $0 \leq i \leq n$, los siguientes diagramas conmutan
\begin{center}
\begin{tikzpicture}
\matrix (m) [matrix of math nodes,row sep=3em,column sep=4em,minimum width=2em]
  {
     X_n & X_{n-1} & & X_n & X_{n+1}\\
     Y_n & Y_{n-1} & & Y_n & Y_{n+1}\\};
  \path[-stealth]
    (m-1-1) edge node [left] {$f_n$} (m-2-1)
    (m-1-4) edge node [left] {$f_n$} (m-2-4)
    (m-1-2) edge node [right] {$f_{n-1}$} (m-2-2)
    (m-1-5) edge node [right] {$f_{n+1}$} (m-2-5)
    (m-1-1) edge node [above] {$d_i$} (m-1-2)
    (m-2-1) edge node [below] {$d_i$} (m-2-2)
    (m-1-4) edge node [above] {$s_i$} (m-1-5)
    (m-2-4) edge node [below] {$s_i$} (m-2-5);
\end{tikzpicture} 
\end{center}

Es decir, un morfismo de conjuntos simpliciales consiste de una colección de aplicaciones que sea compatible con los mapas de caras y degeneraciones.
\end{definition}

\begin{remark} Los conjuntos simpliciales junto con los morfismos antes definidos forman una categoría\footnote{Esta es precisamente la categoría de prehaces de $\ordcat$.}, que notaremos $\cat{sSet}$.
\end{remark}

\begin{remark} Un morfismo $f : K \to L$ de complejos simlpiciales ordenados induce a su vez un morfismo de conjuntos simpliciales dado por $f_n[v_{i_1}, \dots, v_{i_n}] := [f(v_{i_1}), \dots, f(v_{i_n})]$ para cada $n$-símplex $[v_{i_1}, \dots, v_{i_n}]$ (posiblemente degenerado) de $K$.

De forma similar, una función continua $f : X \to Y$ induce un morfismo $f_* : \mathcal{S}(X) \to \mathcal{S}(Y)$ entre conjuntos singulares vía la postcomposición. De hecho, 
\end{remark}

\begin{definition} El \textbf{funtor singular} $\mathcal{S} : \cat{Top} \to \cat{sSet}$ asigna a cada espacio su conjunto singular, y a cada función continua $f : X \to Y$ el morfismo simplicial $f_* : \mathcal{S}(X) \to \mathcal{S}(Y)$ dado por $(f_*)_n(\sigma) =  f \circ \sigma$ para cada $\sigma: |\ss{n}| \to X$.
\end{definition}

\begin{remark} Si $X$ es un conjunto simplicial, el conjunto $X_n$ está determinado por los morfismos de conjuntos simpliciales de $\ss{n}$ a $X$. 

Concretamente, por el lema de Yoneda tenemos una biyección natural
\[
\hom_{\cat{sSet}}(\ss{n},X) \simeq X_n
\]

que a cada elemento $x \in X_n$ le asigna un morfismo de conjuntos simpliciales $\iota_x : \ss{n} \to X$ que satisface $\iota_x(1_{\ord{n}}) = x$. 

Esto se corresponde con la intuición que proveen los ejemplos anteriores, en los que los $n$-símplices de un conjunto simplicial son alguna «manifestación» de el $n$-símplex estándar: como la cara de un $m$-símplex de dimensión mayor, como un $m$-símplex de dimensión menor que represente un colapso del mismo, o como el $n$-símplex singular de un espacio topológico.\\
\end{remark}

\begin{definition} Sea $X$ un conjunto simplicial. Un $n$-simplex $x \in X_n$ se dice \textbf{degenerado} si existe $y \in X_{n-1}$ tal que $s_i(y) = x$ para algún $i \in \natzero{n}$. En caso contrario, decimos que $x$ es \textbf{no degenerado}. 

Notamos $NX_n := \{x \in X_n : \text{ $x$ es no degenerado}\}$ y $NX_{\leq n} = \bigcup_{0 \leq k \leq n}NX_k$ para cada $n \in \N_0$. Definimos también el conjunto $NX := \bigcup_{n \in \N_0}NX_n$ de todos los símplices no degenerados.
\end{definition}

\begin{proposition} Sea $X$ un conjunto simplicial. Si $x \in X$ es un símplex degenerado, entonces existe un único símplex no degenerado $y \in X$ tal que $x = s_{i_1} \cdots s_{i_k}y$.
\end{proposition}
\begin{proof} Como $x$ es degenerado, sabemos que existe $x_1 \in X$ y un mapa de degeneración $s_{i_1}$ tal que $x = s_{i_1}(x_1)$. 

Procediendo inductivamente\footnote{Notemos que el proceso termina pues cada símplex que tomamos tiene una dimensión menos, y un $0$-símplex nunca es degenerado.}, obtenemos una sucesión de mapas de degeneración $s_{i_1},\dots, s_{i_k}$ y un símplex no degenerado $y \in X$ tal que
\[
x = s_{i_1} \cdots s_{i_k}(y).
\]

Para terminar, veamos la unicidad. Supongamos que existen mapas de degeneraciones $s_{j_1}, \dots, s_{j_l}$ y un símplex $z \in X$ tales que
\[
s_{i_1} \cdots s_{i_k}(y) = x = s_{j_1} \cdots s_{j_l}(z).
\]
Componiendo a derecha por $d_{i_k} \cdots d_{i_1}$, es
\[
y = d_{i_k} \cdots d_{i_1}s_{j_1} \cdots s_{j_l}(z).
\]

Usando las identidades simpliciales, podemos «reordenar» los mapas de forma que existen $s$ y $d$, composiciones de mapas de caras y degeneraciones respectivamente, que satisfacen
\[
y = sd(z).
\]
Como $y$ es no degenerado, debe ser $s = 1$ y por lo tanto $y = dz$ es una cara de $z$. Por simetría, vemos también que $z$ es una cara de $y$. En particular, los símplices $z$ e $y$ deben tener la misma dimensión, por lo que debe ser $d = 1$ y $z = y$. 
\end{proof}

\begin{definition} Sea $k \in \N_0$. Un punto $p \in |\ss{k}|$ del $k$-símplex topólogico se dice \textbf{interior} si no existe un mapa de cocara $d^i : \ss{k-1} \to \ss{k}$ y $q \in |\ss{k-1}|$ tal que $|d^i|(q) = p$.
\end{definition}

\begin{proposition} Sea $k \in \N_0$ y $p \in |\ss{k}|$. Si $p$ es interior, existen únicos $l \in \N_0$, $q \in |\ss{l}|$ y $d^{i_1}, \dots, d^{i_s}$ mapas de cocara tales que $p = |d^{i_1} \cdots d^{i_s}|(q)$.
\end{proposition}
\begin{proof} Como $p$ es interior, existe un mapa de cocara $d^{i_1}$ y $q_1 \in |\Delta^{k-1}|$ tal que $d^{i_1}q_1 = p$. Procedemos inductivamente obtenemos la existencia. 

Ahora, supongamos que $q \in |\ss{k}|,q'\in|\ss{k'}|$ son tales que existen mapas de cocara $d^{i_1}, \dots, d^{i_r},d^{j_1}, \dots, d^{j_s}$ con $|d^{i_1}, \dots, d^{i_s}|(q) = |d^{j_1}, \dots, d^{j_s}|(q')$. Usando las identidades cosimpliciales existen funciones $s, \ d$ que son composiciones de codegeneraciones y cocaras respectivamente, y satisfacen
\[
q = |ds|(q').
\]
Como $q$ es interior, debe ser $|d| = 1$ y entonces $q'$ se colapsa a $q$. Por simetría tenemos entonces que ambos puntos se colapsan el uno al otro: esto implica $s = 1$ y $q  = q'$.
\end{proof}

\begin{corollary} Si $p \in |\ss{k+1}|$ es interior, y $s^j$ es un mapa de codegeneración, entonces $s_j(p)$ es interior. 
\end{corollary}
\begin{proof} Notemos que para todo $\sum_{i=1}^{n-1}\alpha_ ie_i \in |\ss{n-1}|$ y mapa de cocara $d^i$, es 
\[
d^iz= \sum_{j=1}^{i-1}\alpha_j e_j + \sum_{j = i+1}^{n}\alpha_{j-1} e_j.
\]
Por lo tanto, un punto interior debe tener todas sus coordenadas no nulas. En consecuencia el punto $p = \sum_{j=1}^{k+1}\alpha_je_j \in |\ss{k+1}|$ debe tener todas sus coordenadas no nulas. Como es
\[
s^j(p) = \sum_{j < i}\alpha_j e_j + (\alpha_i+\alpha_{i+1})e_i + \sum_{j > i+1}^k \alpha_{j}e_{j-1}
\]
que una vez más tiene todas sus coordenadas no nulas, debe ser entonces un punto interior.
\end{proof}

\section{Realización Geométrica}

Durante la materia vimos como a partir de un complejo simplicial $K$ podemos construir un espacio topológico $|K|$, la realización geométrica de $K$. Siguiendo esta idea, queremos extender esta noción al contexto de los conjuntos simpliciales.

\begin{definition} Sea $X$ un conjunto simplicial. Dotando a cada conjunto $X_n$ de la topología discreta, definimos la \textbf{realización geométrica} de $X$ como el espacio
\[
|X| = \left(\coprod_{n \geq 0}X_n \times |\ss{n}|\right)\Big/\sim
\]
donde identificamos a los puntos $(x,|d^i|(p)) \sim (d_i(x),p)$ y $(x,|s^i|(p)) \sim (s_i(x),p)$.
\end{definition}

Esto formaliza la intuición anterior: para cada $x \in X_n$ construimos una copia del $n$-símplex estándar y los pegamos en función de si son caras o colapsos unos de otros.

Además, como veremos en breve esta asignación es funtorial, teniéndose así un análogo a la realización geométrica para complejos simpliciales. Más aún, estas nociones coindicen cuando $X$ es el conjunto simplicial asociado a un complejo simplicial ordenado.

\begin{proposition} La realización geométrica del $n$-símplex estándar $\ss{n} :\ordcat^{op} \to \cat{Set}$ es homeomorfa a la realización geométrica del $n$-símplex estándar como complejo simplicial.
\end{proposition}
\begin{proof} Para evitar ambiguedades, en toda la demostración notaremos $|\ss{n}|$ exclusivamente para referirnos a la realización geométrica del $n$-símplex como complejo simplicial. Por otro lado, notaremos $|\ordcat(-,n)|$.

Para cada $k \in \N_0$ tenemos una aplicación
\[
(f,x) \in \ordcat(k,n) \times |\Delta^k| \to |f|(x) \in |\Delta^n|,
\]
que resulta continua pues cada espacio $\ordcat(k,n)$ es discreto.

Ésta familia de funciones induce un morfismo en el coproducto que pasa al cociente por las identificaciones de la realización geométrica, pues si $g$ es un mapa de cocara o codegeneración, entonces
\[
r(g^*(f),x) = |fg|(x) = r(f,|g|(x)).
\]
Se tiene entonces una función continua $r : [(f,x)] \in |\ordcat(-,n)| \to |f|(x) \in |\Delta^n|$. 

Por otro lado, podemos considerar la inclusión $i : |\ss{n}| \to |\ordcat(-,n)|$ dada por la composición
\[
|\ss{n}| \xrightarrow{\simeq} \{id\} \times |\ss{n}| \hookrightarrow |\ordcat(-,n)|,
\]
que satisface $ri = 1$ pues
\[
ri(x) = [(id,x)] = |id|(x) = x
\]
para todo $x \in |\ss{n}|$. En particular sabemos que $i$ es inyectiva. Por lo tanto, para terminar basta ver que $i$ es sobreyectiva, en cuyo caso es un homeomorfismo con inversa $f$.

Equivalentemente, resta ver que todo elemento $[(f,x)]$ está relacionado con un punto de la forma $(id,y)$ para cierto $y \in |\ss{n}|$. En efecto, sea $(f,x) \in |\ordcat(-,n)|$ con $f : \ord{k} \to \ord{n}$ un morfismo de posets y $x \in |\ss{k}|$. Sabemos entonces que existen mapas de cocara o codegeneración $f_1, \dots, f_n$ tales que $f = f_1 \cdots f_n$. En consecuencia es
\begin{align*}
[(f,x)] &= [(f_1 \cdots f_n, x)] = [(f_n^* \circ \cdots \circ f_1^*(id), x)]\\ &= [(id,|f_1 \cdots f_n|(x))] = [(id, |f|(x))],
\end{align*}
lo que concluye la demostración.
\end{proof}

\begin{proposition} Si $K$ es un complejo simplicial ordenado, entonces sus realizaciones geométricas como conjunto y complexo simplicial coinciden.
\end{proposition}
\begin{proof} Notemos $\mathfrak{K}$ a la realización geométrica de $K$ como conjunto simplicial y $|K|$ a la realización geométrica como complejo simplicial. Para cada símplex generalizado $x  = [v_{i_1}\cdots v_{i_k}]$, notamos $\Delta_x := \{v_{i_1}, \dots, v_{i_k}\}$. De la misma forma, notaremos $[\sigma]$ al símplex ordenado que se corresponde con $\sigma \in K$.

Ahora, tenemos una flecha $|\ss{k}| \to |\Delta_x|$ vía el morfismo de posets que envía a cada $s \in \ord{k}$ a $v_{i_s}$, lo que induce una función continua  $f_x : \{x\} \times |\Delta^k| \to |\Delta_x|$ para cada $x\in X$. A través de la inclusión $|\Delta_x| \hookrightarrow |K|$ obtenemos finalmente funciones continuas $ \{x\} \times |\ss{k}| \to |K|$ para cada $x \in X$. Por un cálculo directo, siempre se tiene $f_{d_i(x)}(p)  = f_x(|d^i|(p))$ y $f_{s_i(x)}(p) = f_x(|s^i|(p))$, así que estas inducen una aplicación continua
\[
f: \mathfrak{K} \to |K|.
\]

Por otro lado, tenemos una función $g : |K| \to \mathfrak{K}$ que envía la combinación convexa (ordenada, con escalares no nulos) $\sum_{i=1}^k a_iv_{i_k}$ a la clase del elemento $([v_{i_1}, \dots, v_{i_k}] ,a_1e_1 + \dots + a_ke_k)$.

La continuidad de $g$ se desprende de que el diagrama
\begin{center}
\begin{tikzcd}
\text{$|K|$} \arrow{r}{g} &\mathfrak{K}\\
\text{$|\sigma|$} \arrow[hook]{u} \arrow{r}{\equiv} & \text{$\{[\sigma]\} \times |\ss{k}|$} \arrow[hook]{u}
\end{tikzcd}
\end{center}
es conmutativo para cada $k$-símplex $\sigma \in K$ y $|K|$ tiene la topología final con respecto a sus símplices. 

Para terminar, veamos que $f$ y $g$ son inversas: por construcción, para cada símplex $\sigma \in K$ se tiene que $f_{[\sigma]}g|_{|\sigma|} = 1$, así que ya sabemos que $fg = 1$. Por otro lado, si $x \in X$ es no degenerado entonces es $[\Delta_x] = x$ y  $g|_{|\Delta_x|}f_x = 1$. Como todo punto de $\mathfrak{K}$ es equivalente a la clase de uno cuyo símplex generalizado es no degenerado, obtenemos finalmente que $gf = 1$.
\end{proof}

Vemos así que la construcción que dimos efectivamente generaliza a la de realización geométrica para complejos simpliciales: tiene sentido preguntarse entonces, ¿qué sucede cuando $X$ no necesariamente viene inducido por un complejo simplicial? Veremos más adelante que el espacio topológico $|X|$ siempre es un CW-complejo.

\begin{definition} Sea $X$ un conjunto simplicial. Un punto $(x,p) \in \coprod_{n \geq 0} X_n \times |\ss{n}|$ se dice \textbf{no degenerado} si $p$ es interior y $x$ no degenerado.
\end{definition}

\begin{lemma}\label{uniq-pt-nodeg} Sea $X$ un conjunto simplicial. Si $(x,p) \in \coprod_{n \geq 0} X_n \times |\ss{n}|$, entonces existe un único punto no degerado $(y,q)$ relacionado con $(x,p)$.
\end{lemma}
\begin{proof} Para cada símplex $x \in X$, sabemos que existe un único símplex no degenerado $y \in X$ y $s = X(\sigma)$ una composición de mapas de degeneración tales que $s(y) = x$. Del mismo modo, existe un único punto $q$ de un símplex topológico y una composición de mapas de cocaras $\delta$ tales que $p = \delta(q)$. Notando $\lambda(x,p) := (y,\sigma p)$ y $\rho(x,p) := (X(\delta)(x),q)$, tenemos definidas dos aplicaciones
\[
\lambda, \rho : \coprod_{n \geq 0} X_n \times |\ss{n}| \to \coprod_{n \geq 0} X_n \times |\ss{n}|
\] 

Más aún, la imagen de $\lambda$ consiste de puntos cuyo símplice de $X$ es no degenerado, y por otro lado la imagen de lambda consiste de pares cuyo elemento del símplex topológico es interior. Por lo tanto, la composición
\[
\Phi := \lambda \circ \rho : \coprod_{n \geq 0} X_n \times |\ss{n}| \to \coprod_{n \geq 0} X_n \times |\ss{n}|
\]
envía cada punto a uno relacionado que es no degenerado, lo que prueba la existencia.

Notemos además que $\Phi$ deja fijos a los puntos no degenerados, así que para ver la unicidad resta probar que puntos relacionados tienen la misma imagen por $\Phi$. De hecho, alcanza probar que  $\Phi(X(d)x,p) = \Phi(x,|d|p)$ y $\Phi(X(s)x,|s|p)$ para cada mapa de cocara $d$ y codegeneración $s$. 

Tomemos un punto $(x,p)$ con $p = |\delta|q$ y $q$ interior. Se tiene entonces que
\[
\rho(X(d)x,p) = \rho(X(d),|\delta|q) = (X(\delta)X(d)x,q) = (X(d\delta)x,q) = \rho(x,|d\delta|q) = \rho(x,|d|p),
\]
para cada mapa $d$ de cocara. En particular debe ser $\Phi(X(d)x,p) = \Phi(x,|d|p)$. 

Si ahora $s$ es un mapa de degeneración, podemos escribir $s\delta = \delta's'$ y $X(\delta')x = X(\sigma)z$ y cierta composición de mapas de cocara $\delta'$ y codegeneración $s',\sigma$, con $z$ no degenerado. Con esta notación es
\[
X(\delta)X(s)x = X(s\delta)x = X(\delta's')x = X(s')X(\delta')x = X(s')X(\sigma)z = X(\sigma s')z.
\]
lo que finalmente dice que
\begin{align*}
\Phi(X(s)x,p) &= \lambda(X(\delta)X(s)x,q) = \lambda(X(\sigma s')z,q) = (z,|\sigma s'|q)\\
&= \lambda(X(\sigma)z,|s'|q) = \lambda(X(\delta')x,|s'|q) = \lambda\rho(x,|\delta's'|q)\\
&= \Phi(x,|s\delta|q) = \Phi(x,|s|p).
\end{align*}
\end{proof}

\begin{theorem} La realización geométrica de un conjunto símplicial $X$ es un CW-complejo, con una celda por cada símplice $x \in X$ no degenerado.
\end{theorem}

\begin{proof} Notemos $q : \coprod_{n \geq 0} X_n \times |\ss{n}| \to |X|$ a la proyección al y $\Delta_x := q(\{x\} \times |\ss{n}|)$ para cada $x \in NX$. Tenemos luego funciones continuas
\[
f_x : p \in |\ss{n}| \mapsto q(x,p) \in \Delta_x \subset |X|
\]
para cada $x \in X$. 

Afirmamos que esto le dá una estructura celular a la realización geométrica, cuyas $n$-celdas son los conjuntos $\Delta_x$ para cada $x \in NX_n$. Así, los $n$-esqueletos resultan
\[
\sk_n |X| := \bigcup_{x \in NX_{\leq n}}\Delta_x
\]
para cada $n \in \N_0$. Por el Lema \ref{uniq-pt-nodeg}, sabemos que $\bigcup_n \sk_n|X| = |X|$, y que dado $(x,p) \sim (y,q)$ con $x \in X_n, y \in X_m$ e $(y,q)$ no degenerado implica $n \geq m$. De esto ultimo vemos que si $x \in NX_n$, entonces
\[
\Delta_x \cap \sk_{n-1}|X| = \{[(x,p)] : p \in \partial|\ss{n}|\}.
\]

En efecto, si $p \in \partial|\ss{n}|$ debe ser $p = |d^i|(q)$, y por lo tanto $(x,p) \sim (d_i(x),q)$ es equivalente a algún punto no degenerado cuya primera coordenada debe pertenecer a $NX_{\leq n-1}$. Reciprocamente, si $p \not \in \partial|\ss{n}|$ entonces $(x,p)$ es no degenerado, y en consecuencia no está relacionado con ningún elemento del $(n-1)$-esqueleto.

De la caracterización de $\dot{\Delta_x}$ se sigue que $\tint{\Delta_x} = \{ [(x,p)] : p \in \tint{|\ss{n}|}\}$ y por lo tanto cada función $f_x$ es biyectiva, continua, y la restricción $f_x : \partial|\ss{n}| \to \dot{\Delta_x}$ es sobreyectiva. Para concluir que se tiene una estructura celular en $|X|$ debemos ver entonces que $f_x$ es abierta al restringirla al interior de $\Delta_x$.  $\tpaint{[FALTA TERMINAR (Y REVISAR)]}$
\end{proof}


\begin{definition} Sea $X$ un conjunto simplicial. Su \textbf{categoría de símplices} es la categoría coma $\Delta \downarrow X$, donde $\Delta: \ordcat \to \cat{Set}^{\ordcat^{op}}$ es el embedding de Yoneda de $\ordcat$ y $X : \ordcat \to \cat{Set}^{\ordcat^{op}}$ es el funtor que vale constantemente $X$. Concretamente, los objetos de $\Delta \downarrow X$ son \textit{símplices} de $X$, entendidos como morfismos simpliciales $\ss{n} \to X$, y las flechas son morfismos $\theta : \ord{n} \to \ord{m}$ de posets tales que el diagrama
\begin{center}
\begin{tikzpicture}
\matrix (m) [matrix of math nodes,row sep=3em,column sep=4em,minimum width=2em]
  {
     \ss{n}& &\\
     & & X\\
     \ss{m} & &\\};
  \path[-stealth]
    (m-1-1) edge node [above] {$\sigma$} (m-2-3)
    (m-1-1) edge node [left] {$\theta_*$} (m-3-1)
    (m-3-1) edge node [below] {$\tau$} (m-2-3);
\end{tikzpicture} 
\end{center}
conmuta, donde $\theta_*$ es la postcomposición por $\theta$.
\end{definition}

\begin{theorem} Si $X$ es un conjunto simplicial, entonces
\[
|X| \simeq \catcolim{\substack{\ss{n} \to X \\ \text{ en $\Delta \downarrow X$}}}{|\ss{n}|}.
\]
\end{theorem}
\begin{proof} Observemos que si $Z$ es un espacio topológico arbitrario, una función continua $g : |X| \to Z$ se corresponde unívocamente con una función continua $\tilde{g} : \coprod_{n \geq 0} X_n \times |\ss{n}| \to Z$ que sea compatible con la relación que identifica caras y colapsos.

A su vez, usando la propiedad universal del coproducto y el hecho de que cada espacio $X_n$ tiene la topología discreta, esto equivale a dar funciones
\[
\lambda_x : |\ss{n}| \to Z
\]
para cada $x \in X_n$ y $n \geq 0$, que satisfagan las condiciones de compatibilidad de antes: esto es, que para cada morfismo de posets\footnote{Si bien la condición original era sobre los mapas de caras y degeneración, que son imagen por $X$ de los mapas de cocaras y codegeneración en $\ordcat$. Al clausurar la relación de equivalencia se tiene una condición equivalente reemplazando éstos por cualquier función $X(\theta)$ con $\theta$ un morfismo en $\ordcat$.} $\theta : \ord{n} \to \ord{m}$ se tenga $\lambda_{X(\theta)(x)} = \lambda_x \circ |\theta| = |\theta^*|(\lambda_x)$.

Recordemos ahora que, por el lema de Yoneda, tenemos que
\[
\hom(\ss{n},\ss{m}) \simeq \ordcat(n,m) = \{\theta : \ord{n} \to \ord{m} : \text{ $\theta$ es morfismo de posets}\}
\]
y
\[
\hom(\ss{n}, X) \simeq X_n
\]
para cada $n,m \geq 0$. Por lo tanto, cada elemento $x \in X_n$ se corresponde a un único morfismo simplicial $\sigma_x : \ss{n} \to X$, y toda flecha $\ss{n} \to \ss{m}$ es la poscomposición por cierto morfismo de posets $\theta: \ord{n} \to \ord{m}$.

En éstos terminos, la información anterior se puede describir como un morfismo $\lambda_\sigma : |\ss{n}| \to Z$ para cada símplex $\sigma : \ss{n} \to X$, de forma que para todo morfismo de posets $\theta : \ord{n} \to \ord{m}$ y símplices $\sigma : \ss{n} \to X, \tau : \ss{m} \to X$ tales que $\sigma = \tau \  \theta_*$ se tenga que $\lambda_\sigma = \lambda_\tau |\theta|$. 

Es decir, dar un morfismo $g : |X| \to Z$ es equivalente a dar un cocono sobre $Z$ para el funtor $F : \Delta \downarrow X \to \mathsf{Top}$ que envía $(\sigma : \ss{n} \to X) \xrightarrow{\theta_*} (\tau : \ss{m} \to X)$ a $|\ss{n}| \xrightarrow{|\theta|} |\ss{m}|$.

Por otro lado, tenemos un cocono sobre $X$ dado por las funciones
\[
\iota_\sigma : p \in |\ss{n}| \to [(x,p)] \in |X|,
\]
para cada $\sigma : \ss{n} \to X$ que está en correspondencia con $x \in X_n$. Por las observaciones anteriores, sabemos que $(\lambda_\sigma)_{\sigma \in \Delta \downarrow X}$ se factoriza por $(\iota_\sigma)_{\sigma \in \Delta \downarrow X}$ a través de $g$ ya que es
\[
\lambda_\sigma(p) = g([(x,p)]) = g(\iota_\sigma(x,p))
\]
para todo $\sigma : \ss{n} \to X$ y $p \in |\ss{n}|$. Pero como notamos anteriormente, de existir una función continua $|X| \to Z$ está determinada por lo que vale en la imagen de cada morfismo $\iota_\sigma$. 
Hemos visto entonces que todo cono sobre $F : \Delta \downarrow X \to \mathsf{Top}$ se factoriza a través de $(\iota_\sigma)_\sigma$ de forma única, y esto es precisamente que
\[
|X| \simeq \catcolim{\substack{\ss{n} \to X \\ \text{ en $\Delta \downarrow X$}}}{|\ss{n}|}.
\]
\end{proof}

\begin{proposition} Se tiene un funtor
\[
| \cdot | : \cat{sSet} \to \cat{Top}
\]
que asigna a cada conjunto simplicial $X$ su realización geométrica, y a cada morfismo simplicial $f : X \to Y$ una flecha $|f| : |X| \to |Y|$ inducida por cada función $f_n \times 1 : X_n \times |\ss{n}| \to Y_n \times |\ss{n}|$.
\end{proposition}
\begin{proof} En primer lugar, notemos que las aplicaciones $\{f_n \times 1\}_{n \geq 1}$ inducen una función continua
\begin{center}
\begin{tikzcd}
 \coprod_{n \geq 0} X_n \times |\ss{n}| \arrow{r}{\widetilde{f}} &  \coprod_{n \geq 0} Y_n \times |\ss{n}|\\
 X_n \times |\ss{n}| \arrow[hookrightarrow]{u} \arrow{r}{f_n \times 1} & Y_n \times |\ss{n}| \arrow[hookrightarrow]{u}
\end{tikzcd}
\end{center}

entre los coproductos. Como $f$ es un morfismo simplicial, si $\theta : \ord{m} \to \ord{m}$ es un morfismo de posets entonces el diagrama
\begin{center}
\begin{tikzcd}
 X_n \times |\ss{n}| \arrow{r}{X\theta}\arrow{d}[left]{f_n \times 1} &  X_m \times |\ss{n}| \arrow{d}{f_m \times 1}\\
 Y_n \times |\ss{n}| \arrow{r}{Y \theta} & Y_m \times |\ss{n}| 
\end{tikzcd}
\end{center}
conmuta. Por lo tanto, se tiene que
\[
\widetilde{f}(X\theta(x),p) = (f_mX\theta(x),p) = (Y\theta f_n(x),p)
\]
y
\[
\widetilde{f}(x,|\theta|(p)) = (f_n(x),|\theta|(p)),
\]
lo que nos dice que $\widetilde{f}$ manda puntos relacionados en puntos relacionados. En consecuencia, está bien definida la función continua $|f| : |X| \to |Y|$ que envía $[(x,p)]$ a $[(f_n(x),p)]$ si $x \in X_n$, y de esta caracterización se deduce la funtorialidad de $|\cdot|$.
\end{proof}

\section{La adjunción entre la realización geométrica y el funtor singular}

Como vimos en la sección anterior, la realización geométrica está directamente relacionada con el conjunto singular de un espacio topológico. El teorema que sigue describe explícitamente esta relación en terminos de los functores $| \cdot |$ y $\mathcal{S}$.

\begin{theorem} Existe una adjunción
\begin{center}
\begin{tikzcd}
    \cat{Top} \arrow[bend left=35]{rr}{| \  \cdot \ |}
  & \rotatebox{90}{$\vdash$}
  & \cat{sSet} \arrow[bend left=35]{ll}{\mathcal{S}}
\end{tikzcd}
\end{center}
En otras palabras, se tiene una biyección
\[
\cat{Top}(|X|,Y) \simeq \cat{sSet}(X,\mathcal{S}(Y))
\]
entre las funciones continuas $|X| \to Y$ y los morfismos simpliciales $X \to \mathcal{S}(Y)$. Más aún, esta biyección es natural tanto en $X$ como en $Y$.
\end{theorem}
\begin{proof} Recordemos que hay una correspondencia biyectiva entre morfimos simpliciales $\sigma: \ss{n} \to X$ y símplices $x \in X_n$. Notamos $\sigma_x$ al morfismo simplicial asociado a un símplex $x$. Vimos también que una función continua $g : |X| \to Y$ se corresponde unívocamente con una colección de funciones continuas $g_x : \ss{n_x} \to Y$ tales que 
\[
g([(x,p)]) = g_x(p)
\]
para todo símplex $x \in X_n$ y $p \in |\ss{n}|$. 

En vista de esto, definimos
\begin{align*}
e : \cat{Top}(|X|,&Y) \to \cat{sSet}(X,\mathcal{S}(Y))\\
& g \longmapsto e(g)(x) := g_x
\end{align*}
Como $g$ es un morfismo cuyo dominio es $|X|$, las funciones $g_x$ deben satisfacer
\[
g_{X\theta(x)} = |\theta|^*(g_x)
\]
para todo símplice $x$ y morfismo de posets $\theta$. En particular, esto prueba que $e(g)$ es efectivamente un morfismo de complejos simpliciales, por lo que $e$ está bien definida.

Por otro lado, definimos otra aplicación
\begin{align*}
r : \cat{sSet}(X,\mathcal{S}(&Y)) \to \cat{Top}(|X|,Y)\\
& f \longmapsto r(f)
\end{align*}
del siguiente modo: dado un morfismo simplicial $f$, para cada $f_n : X_n \to \mathcal{S}(Y)_n$ definimos las funciones
\[
\widetilde{f_n} : (x,p) \in X_n \times |\ss{n}| \mapsto f_n(x)(p) \in Y,
\]
las cuales inducen a su vez una función continua
\[
\coprod_{n \geq 0} \widetilde{f_n} : \coprod_{n \geq 0} X_n \times |\ss{n}| \to Y.
\]

Como $f$ es un morfismo simplicial, para cada $\theta \in \ordcat(m,n)$ se tiene que
\[
\widetilde{f_m}(X\theta(x), p) = (f_mX\theta)(x)(p) = |\theta|^*(f_n(x))(p) = (f_n(x) |\theta|)(p) = \widetilde{f_n}(x,|\theta|(p)).
\]
Esto dice que la función $\coprod_{n \geq 0} \widetilde{f_n}$ pasa al cociente por las identificaciones de la realización geométrica, y por lo tanto, induce una aplicación continua $|X| \to Y$ que tomamos como imagen de $f$ por $r$. Concretamente, dados $x \in X_n$ y $p \in |\ss{n}|$ es
\[
r(f)([(x,p)]) = f_n(x)(p).
\]

Veamos ahoar que $e$ y $r$ son aplicaciones inversas. Si tomamos $f : X \to \mathcal{S}(Y)$ simplicial, es
\[
er(f)(x)(p) = r(f)([(x,p)]) = f_n(x)(p)
\]
para todo $x \in X_n$ y $p \in |\ss{n}|$. Por lo tanto, se tiene que $e r(f) = f$ y, en general, $e r = 1$.

Recíprocamente, si $g : |X| \to Y$ es una función continua, entonces para cada $(x,p) \in X_n \times |\ss{n}|$ es
\[
r e(g)([(x,p)]) = e(g)_n(x)(p) = g_x(p) = g([(x,p)]),
\]
lo cual prueba que $r e  = 1$.

Para terminar, veamos que el isomorfismo es natural tanto en $X$ como $Y$. Fijemos $g : |X| \to Y$ continua y $(x,p) \in X_n \times |\ss{n}|$. Dado un morfismo simplicial $f : X' \to X$, debemos ver que el diagrama
\begin{center}
\begin{tikzcd}
\cat{Top}(|X|,Y) \arrow{r}{e}\arrow{d}[left]{|f|^*} \  &\ \cat{sSet}(X,\mathcal{S}(Y))\arrow{d}{f^*}\\ 
\cat{Top}(|X'|,Y)\arrow{r}{e} & \cat{sSet}(X',\mathcal{S}(Y)) 
\end{tikzcd}
\end{center}
conmuta. Por un cálculo directo es
\begin{align*}
f^*(e(g))(x)(p) &= e(g)(f(x))(p) = g([(f(x),p)])\\
& = g|f|([(x,p)]) = e(g|f|)(x)(p)\\
& = e(|f|^*(g))(x)(p),
\end{align*}
y por lo tanto $f^*e = e|f|^*$. Del mismo modo, si $f : Y \to Y'$ es continua entonces
\begin{center}
\begin{tikzcd}
\cat{Top}(|X|,Y) \arrow{r}{e}\arrow{d}[left]{f_*} \  &\ \cat{sSet}(X,\mathcal{S}(Y))\arrow{d}{f_*}\\ 
\cat{Top}(|X|,Y')\arrow{r}{e} & \cat{sSet}(X,\mathcal{S}(Y')) 
\end{tikzcd}
\end{center}
conmuta pues
\begin{align*}
f_*(e(g))(x)(p) &= f(e(g)(x)(p)) = fg([(x,p)])\\
&= e(fg)(x)(p) = e(f_*(g))(x)(p).
\end{align*}
\end{proof}

Recuerdo ahora el siguiente resultado general sobre funtores adjuntos:

\begin{theorem} Sean $F : \mathscr{C} \to \mathscr{D}$ y $G : \mathscr{D} \to \mathscr{C}$ dos funtores. Si $F \dashv G$, entonces $G$ preserva límites y $F$ preserva colíimtes. 
\end{theorem}
\begin{proof} Ver \cite{ct-context}, teoremas 4.5.2 y 4.5.3.
\end{proof}

\begin{corollary} La realización geométrica $| \cdot | : \cat{sSet} \to \cat{Top}$ preserva colímites. En particular, preserva coproductos y pushouts. \qed
\end{corollary}

\begin{thebibliography}{}
\bibitem{GJ} P. Goerss y J. Jardine, \textit{Simplicial Homotopy Theory}. Birkhäuser, 2010.
\bibitem{F} G. Friedman. \textit{An elementary introduction to simplicial sets}, arXiv:0809.4221v5 [at], 2016.
\bibitem{ct-context} E. Riehl. \textit{Category Theory in Context}. Dover, 2016.
\end{thebibliography}
\end{document}


